%\section{Exercise C -- Controller Transfer Function}

\section{Available controller functions}

\subsection{PI}
\[
	G(s)
	= K_p \left(1 + \dfrac{\omega_i}{s} \right)
\]

\subsection{PIT1}
\[
	G(s)
	= K_p \dfrac{
		\left(1 + \dfrac{\omega_{1}}{s}\right)
	}{ 
		\left(1 + \dfrac{s}{\omega_{T}} \right)
	}
\]
\subsection{PIDT2}

\[
	G(s)
	= K_p \dfrac{
		\left(1 + \dfrac{\omega_{i1}}{s}\right) \left(1 + \dfrac{\omega_{s}}{\omega_{i2}}\right)
	}{ 
		\left(1 + \dfrac{s}{\omega_{T2}} \right) \left(1 + \dfrac{s}{\omega_{T1}} \right)
	}
\]

\section{Selection of the controller type}
Looking at the bode plot of the derived transfer function of the power stage
we can see, that a simple PI controller is sufficient to get a stable system.

Using the PI controller we can shift the gain down, so that the crossover
frequency can be shifted to a lower frequency. The disadvantage of the PI
controller is, that we'll have a worse starting angle of the phase near the
phase margin of $\varphi_R$ = \SI{\angfmargin}{\degree}, since the I part
introduces a \SI{-90}{\degree} phase shift. Therefore, we have to shift the
crossover frequency far below $\frac{f_s}{2}$ to achieve the specified
phase margin. This will result in lower performance of the controller in terms
of control dynamics.

The use of a more complex controller, like a PIDT2 with differential behaviour,
tends to be more difficult to implement. This is especially true when 
discontinuous conduction or very high duty cycles have to be considered as well,
where the transfer function changes or non-ideal circuitry has a significant
effect on the control loop.

\clearpage
\section{Choosing the controller parameters}
For a simple PI controller the parameters will be choosen empirically by
inspecting the resulting open loop bode diagram and the step response of the
closed loop.

\begin{figure}[h!]
	\centering
	\def\svgscale{0.75}
	\input{../fig/sch/control_loop_open_01.pdf_tex}
	\caption{Open loop control diagram.}
\end{figure}

\begin{figure}[h!]
	\centering
	\def\svgscale{0.75}
	\input{../fig/sch/control_loop_closed_01.pdf_tex}
	\caption{Closed loop control diagram.}
\end{figure}

The empirically obtained parameters without significant overshoot are the following.

\[ K_p = 9 \cdot 10^{-6} \]
\[ \omega_i = 200 \cdot 10^{3} \]

\begin{figure}[h!]
	\begin{subfigure}[b]{0.9\textwidth}
		\centering
		\begin{tikzpicture}
			\begin{axis}[
				%xmode=log,
				%ymode=log,
				%log basis y=10,
				%yticklabel={\pgfmathparse{20*(\tick)}\pgfmathprintnumber[fixed]{\pgfmathresult}},
				height=0.45\textwidth,
				title=\textbf{Closed Loop Step Response},
				width=\linewidth,
				grid=major,
				grid style={dashed,gray!30},
				xlabel={time $t \left[\si{\milli\second}\right]$},
				ylabel={amplitude $\left[\si{1}\right]$},
				x label style={at={(axis description cs:0.5,-0.075)},anchor=north},
				legend style={at={(0.98,0.02)},anchor=south east},
				x tick label style={rotate=90,anchor=east},
				legend columns=1,
				ymin=-0.2,
				ymax=1.2,
				xmin=-1,
				xmax=10
				]
				\addplot+[mark=none] table[x=time, y=signal, col sep=comma]{./../../data/step1.csv};
				\addlegendentry{D = 0.75};
				\addplot+[mark=none] table[x=time, y=signal, col sep=comma]{./../../data/step2.csv};
				\addlegendentry{D = 0.50};
				\addplot+[mark=none] table[x=time, y=signal, col sep=comma]{./../../data/step3.csv};
				\addlegendentry{D = 0.25};
				\draw[dashed] (axis cs: -1,0) -- (axis cs: 0,0) -- (axis cs: 0,1) node[above right]{ideal} -- (axis cs: 10,1);
			\end{axis}
		\end{tikzpicture}
	\end{subfigure}

\end{figure}


\clearpage
\section{Bode diagram of the control function}
\begin{figure}[h!]
	\begin{subfigure}[b]{0.9\textwidth}
		\centering
		\begin{tikzpicture}
			\begin{axis}[
				xmode=log,
				ymode=log,
				log basis y=10,
				yticklabel={\pgfmathparse{20*(\tick)}\pgfmathprintnumber[fixed]{\pgfmathresult}},
				height=0.45\textwidth,
				title=\textbf{Controller Transfer Function -- Amplitude Response},
				width=\linewidth,
				grid=minor,
				grid style={dashed,gray!30},
				xlabel={angular frequency $\omega \left[\si{\per\second}\right]$},
				ylabel={gain $\left|G\right| \left[\textrm{dB}\right]$},
				x label style={at={(axis description cs:0.5,-0.075)},anchor=north},
				legend style={at={(0.99,0.98)},anchor=north east},
				x tick label style={rotate=90,anchor=east},
				legend columns=1,
				ymin=0.000001,
				ymax=0.01,
				xmin=\angflowerbound,
				xmax=\angfupperbound
				]
				\addplot+[mark=none] table[x=angfreq, y=magnitude, col sep=comma]{./../../data/g.csv};
				\addlegendentry{$G$};
			\end{axis}
		\end{tikzpicture}
	\end{subfigure}

	\begin{subfigure}[b]{0.9\textwidth}
		\centering
		\begin{tikzpicture}
			\begin{axis}[
				xmode=log,
				ymode=linear,
				height=0.45\textwidth,
				title=\textbf{Controller Transfer Function -- Phase Response},
				width=\linewidth,
				grid=minor,
				grid style={dashed,gray!30},
				xlabel={angular frequency $\omega \left[\si{\per\second}\right]$},
				ylabel={phase $\varphi \left[\si{\degree}\right]$},
				x label style={at={(axis description cs:0.5,-0.075)},anchor=north},
				legend style={at={(0.99,0.98)},anchor=north east},
				x tick label style={rotate=90,anchor=east},
				legend columns=1,
				ymin=\phaselowerbound,
				ymax=\phaseupperbound,
				xmin=\angflowerbound,
				xmax=\angfupperbound
				]
				\addplot+[mark=none] table[x=angfreq, y=phase, col sep=comma]{./../../data/g.csv};
				\addlegendentry{$G$};
				\draw[dashed] (axis cs: \angflowerbound,\angfmargin) node[below right]{$\varphi_{R}$ = \SI{-120}{\degree}} -- (axis cs: \angfupperbound,\angfmargin);

			\end{axis}
		\end{tikzpicture}
	\end{subfigure}
\end{figure}

By inspecting the bode plot of the controller transfer function, one can see,
that the suggested controller is essentially an I controller, since the
corner frequency is very close to the limit of $\frac{f_s}{2}$
($\omega = \pi f_s$ in the plot). Therefore, the designed PI controller is
essentially working as an I controller within the defined range of the power
stage transfer function.
