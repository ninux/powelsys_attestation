\section{Exercise C -- Controller Transfer Function}

\subsection{Available controller functions}

\subsubsection{PI}
\[
	G
	= K_p \left(1 + \dfrac{K_i}{s} \right)
\]

\subsubsection{PIT1}
\[
	G
	= K_p \dfrac{
		\left(1 + \dfrac{\omega_{1}}{s}\right)
	}{ 
		\left(1 + \dfrac{s}{\omega_{T}} \right)
	}
\]
\subsubsection{PIDT2}

\[
	G
	= K_p \dfrac{
		\left(1 + \dfrac{\omega_{i1}}{s}\right) \left(1 + \dfrac{\omega_{s}}{\omega_{i2}}\right)
	}{ 
		\left(1 + \dfrac{s}{\omega_{T2}} \right) \left(1 + \dfrac{s}{\omega_{T1}} \right)
	}
\]

\subsection{Selection of the controller type}
Looking at the bode plot of the derived transfer function of the power stage
we can see, that a simple PI controller is sufficient to get a stable system.

Using the PI controller we can shift the gain down, so that the crossover
frequency can be shifted to a lower frequency. The disadvantage of the PI
controller is, that we'll have a worse starting angle of the phase near the
phase margin of $\varphi_R$ = \SI{\angfmargin}{\degree}. Therefore, we have to
shift the crossover frequency far below $\frac{f_s}{2}$ to achieve the specified
phase margin. This will result in lower performance of the controller in terms of
control speed.

The use of a more complex controller, like a PIDT2 with differential behaviour,
tends to be more difficult to implement. This is especially true when 
discontinuous conduction or very high duty cycles have to be considered as well,
where the transfer function changes or non-ideal circuitry has a significant
effect on the control.

\subsection{Choosing the controller parameters}
For a simple PI controller the parameters will be choosen empirically by
inspecting the resulting open loop bode diagram and the step response of the
closed loop.

\subsubsection{Parameters of the control function}
The empirically obtained parameters that work well are the following.

\[ K_p = 0.0001 \]
\[ K_i = 20000 \]

\subsubsection{Bode diagram of the control function}
\begin{figure}[h!]
	\begin{subfigure}[b]{0.9\textwidth}
		\centering
		\begin{tikzpicture}
			\begin{axis}[
				xmode=log,
				ymode=log,
				log basis y=10,
				yticklabel={\pgfmathparse{20*(\tick)}\pgfmathprintnumber[fixed]{\pgfmathresult}},
				height=0.45\textwidth,
				title=\textbf{Gain},
				width=\linewidth,
				grid=minor,
				grid style={dashed,gray!30},
				xlabel={$\omega$},
				ylabel={$\left|G\right|$},
				x label style={at={(axis description cs:0.5,-0.075)},anchor=north},
				legend style={at={(0.99,0.98)},anchor=north east},
				x tick label style={rotate=90,anchor=east},
				legend columns=1,
				ymin=0.000001,
				ymax=0.01,
				xmin=\angflowerbound,
				xmax=\angfupperbound
				]
				\addplot+[mark=none] table[x=angfreq, y=magnitude, col sep=comma]{./../../data/g.csv};
				\addlegendentry{$H_1$};
				\addplot+[mark=none] table[x=angfreq, y=magnitude, col sep=comma]{./../../data/g.csv};
				\addlegendentry{$H_2$};
				\addplot+[mark=none] table[x=angfreq, y=magnitude, col sep=comma]{./../../data/g.csv};
				\addlegendentry{$H_3$};
			\end{axis}
		\end{tikzpicture}
	\end{subfigure}

	\begin{subfigure}[b]{0.9\textwidth}
		\centering
		\begin{tikzpicture}
			\begin{axis}[
				xmode=log,
				ymode=linear,
				height=0.45\textwidth,
				title=\textbf{$H_1$ Phase},
				width=\linewidth,
				grid=minor,
				grid style={dashed,gray!30},
				xlabel={$\omega$},
				ylabel={$\varphi$},
				x label style={at={(axis description cs:0.5,-0.075)},anchor=north},
				legend style={at={(0.99,0.98)},anchor=north east},
				x tick label style={rotate=90,anchor=east},
				legend columns=1,
				ymin=\phaselowerbound,
				ymax=\phaseupperbound,
				xmin=\angflowerbound,
				xmax=\angfupperbound
				]
				\addplot+[mark=none] table[x=angfreq, y=phase, col sep=comma]{./../../data/g.csv};
				\addlegendentry{$H_1$};
				\addplot+[mark=none] table[x=angfreq, y=phase, col sep=comma]{./../../data/g.csv};
				\addlegendentry{$H_2$};
				\addplot+[mark=none] table[x=angfreq, y=phase, col sep=comma]{./../../data/g.csv};
				\addlegendentry{$H_3$};
				\draw[dashed] (axis cs: \angflowerbound,\angfmargin) node[below right]{$\varphi_{R}$ = \SI{-120}{\degree}} -- (axis cs: \angfupperbound,\angfmargin);

			\end{axis}
		\end{tikzpicture}
	\end{subfigure}
\end{figure}
