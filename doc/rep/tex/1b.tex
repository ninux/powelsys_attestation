%\section{Exercise B -- Power Stage Transfer Function}

\section{Differential Equations}
To evaluate the system matrices we have to find expressions for the state
variables, which represent the stored energy. These will be separately
defined for the first (active) and second (passive) interval of the
boost converter. Since the boost converter has two energy storages,
an inductor and a capacitor, we have to find expressions for both of
them for both of the intervals.

\subsection{First Interval}

\subsubsection{Inductor energy}
\[ 
	v_L(t)
	= L \frac{d\,i_L(t)}{d\,t}
\]
\[
	\frac{d\,i_L(t)}{d\,t}
	= \frac{v_L(t)}{L}
	= \frac{v_i(t)}{L}
	\quad, v_L(t) = v_i(t)
\]

\subsubsection{Capacitor energy}
\[
	i_C(t) 
	= C \frac{d\,v_C(t)}{d\,t}
\]
\[
	\frac{d\,v_C(t)}{d\,t}
	= \frac{i_C(t)}{C}
	= \frac{-i_O(t)}{C}
	= \frac{-v_O(t)}{R C}
	= \frac{-v_C(t)}{R C}
\]

\subsubsection{System matrix}
\[
	x' = A_1 x + B_1
\]
\[
	%\begin{bmatrix}
	%	\frac{d\,i_L(t)}{d\,t} \\
	%	\frac{d\,v_C(t)}{d\,t}
	%\end{bmatrix}
	%=
	\begin{bmatrix}
		{i_L}' \\
		{v_C}'
	\end{bmatrix}
	=
	\begin{bmatrix}
		0 	& 0 \\
		0	& \frac{1}{R C}
	\end{bmatrix}
	\cdot
	\begin{bmatrix}
		i_L \\
		v_C
	\end{bmatrix}
	+
	\begin{bmatrix}
		\frac{v_i}{L} \\
		0
	\end{bmatrix}
\]

\subsection{Second Interval}

\subsubsection{Inductor energy}
\[ 
	v_L(t)
	= L \frac{d\,i_L(t)}{d\,t}
\]
\[
	\frac{d\,i_L(t)}{d\,t}
	= \frac{v_L(t)}{L}
	= \frac{v_i(t) - v_o(t)}{L}
	= \frac{v_i(t) - v_c(t)}{L}
	\quad, v_o(t) = v_c(t)
\]

\subsubsection{Capacitor energy}
\[
	i_C(t) 
	= C \frac{d\,v_C(t)}{d\,t}
\]
\[
	\frac{d\,v_C(t)}{d\,t}
	= \frac{i_C(t)}{C}
	= \frac{i_L(t) - \frac{v_C(t)}{R}}{C}
	= \frac{i_L(t)}{C} - \frac{v_C(t)}{R C}
\]

\subsubsection{System matrix}
\[
	x'  = A_2 x + B_2
\]
\[
	%\begin{bmatrix}
	%	\frac{d\,i_L(t)}{d\,t} \\
	%	\frac{d\,v_C(t)}{d\,t}
	%\end{bmatrix}
	%=
	\begin{bmatrix}
		{i_L}' \\
		{v_C}'
	\end{bmatrix}
	=
	\begin{bmatrix}
		0 		& \frac{-1}{L} \\
		\frac{1}{C}	& \frac{-1}{R C}
	\end{bmatrix}
	\cdot
	\begin{bmatrix}
		i_L \\
		v_C
	\end{bmatrix}
	+
	\begin{bmatrix}
		\frac{v_i}{L} \\
		0
	\end{bmatrix}
\]

\section{Complete System Equation}

\[
	x' = \underbrace{(A_1 D + A_2 D')}_{A} x + \underbrace{B_1 D + B_2 D'}_{B} \quad, D + D' = 1
\]

\subsection{System matrix}
\[
	A
	=
	\begin{bmatrix}
		0 & 0 \\
		0 & \frac{-1}{R C}
	\end{bmatrix}
	\cdot D
	+
	\begin{bmatrix}
		0 		& \frac{-1}{L} \\
		\frac{1}{C} 	& \frac{-1}{R C}
	\end{bmatrix}
	\cdot D'
	=
	\begin{bmatrix}
		0		& 0 \\
		0		& \frac{-D}{R C}
	\end{bmatrix}
	+
	\begin{bmatrix}
		0		& \frac{-D'}{L} \\
		\frac{D'}{C}	& \frac{-D'}{R C}
	\end{bmatrix}
	=
	\begin{bmatrix}
		0		& \frac{-D'}{C} \\
		\frac{D'}{C}	& \frac{-D-D'}{R C}
	\end{bmatrix}
	=
	\begin{bmatrix}
		0		& \frac{-D'}{L} \\
		\frac{D'}{C}	& \frac{-1}{R C}
	\end{bmatrix}
\]

\subsection{Source vector}
\[
	B
	= 
	\begin{bmatrix}
		\frac{v_i(t)}{L} \\
		0
	\end{bmatrix}
	\cdot D
	+
	\begin{bmatrix}
		\frac{v_i(t)}{L} \\
		0
	\end{bmatrix}
	\cdot D'
	=
	\begin{bmatrix}
		\frac{V_i D}{L} \\
		0
	\end{bmatrix}
	+
	\begin{bmatrix}
		\frac{V_i D'}{L} \\
		0
	\end{bmatrix}
	=
	\begin{bmatrix}
		\frac{V_i D + V_i D'}{L} \\
		0
	\end{bmatrix}
	=
	\begin{bmatrix}
		\frac{V_i(D+D'}{L} \\
		0
	\end{bmatrix}
	=
	\begin{bmatrix}
		\frac{V_i(D + 1 - D}{L} \\
		0
	\end{bmatrix}
	=
	\begin{bmatrix}
		\frac{v_i(t)}{L} \\
		0
	\end{bmatrix}
\]

\section{Steady State Equation}

\[ x' = 0 \]
\[ 0 = AX + B \]
\[ -AX = B \]
\[ X = -\frac{1}{A}B \]
\[ X = -A^{-1}B \]

\[
	X
	=
	\begin{bmatrix}
		I_L \\
		V_C
	\end{bmatrix}
	=
	\begin{bmatrix}
		\frac{v_i(t)}{D'^2 R} \\
		\frac{v_i(t)}{D'}
	\end{bmatrix}
\]

\[ V_O = F X \]
\[
	V_O
	= 
	\begin{bmatrix}
		0 & 1
	\end{bmatrix}
	\cdot
	\begin{bmatrix}
		\frac{v_i(t)}{D'^2 R} \\
		\frac{v_i(t)}{D'}
	\end{bmatrix}
\]
\[
	I_L
	= \frac{V_i}{D'^2 R}
	= \frac{V_i}{(1-D)^2 R}
\]

\section{Small Signal Pertubation}
\[	E = (A_1 - A_2)X + B1 - B2 \]
\[
	E =
	\begin{bmatrix}
		0 & \frac{1}{L} \\
		\frac{-1}{C} & 0
	\end{bmatrix}
	\cdot
	\begin{bmatrix}
		I_L \\
		V_C
	\end{bmatrix}
	+ \underbrace{
	\begin{bmatrix}
		\frac{V_i}{L} \\
		0
	\end{bmatrix}
	-
	\begin{bmatrix}
		\frac{V_i}{L} \\
		0
	\end{bmatrix}}_{0}
	\quad, B_1 = B_2
\]

\[
	\hat{V_O}(s) = F\hat{X}(s)
\]

\[
	H(s)
	= \frac{\hat{V_O}(s)}{\hat{D}(s)}
	= F ( s I - A )^{-1} E
\]

\[
	\frac{\hat{V_O}(s)}{\hat{D}(s)}
	=
	\begin{bmatrix}
		0 & 1
	\end{bmatrix}
	\left( s 
	\begin{bmatrix}
		1 & 0 \\
		0 & 1
	\end{bmatrix}
	-
	\begin{bmatrix}
		0 & \frac{-D'}{L} \\
		\frac{D'}{C} & \frac{-1}{R C}
	\end{bmatrix}
	\right)^{-1}
	\begin{bmatrix}
		\frac{V_C}{L} \\
		\frac{-I_L}{C}
	\end{bmatrix}
\]

\[
	H(s) = \frac{-R(L I_L s - D' V_C)}{R(C L s^2 + D'^2) + L s}
\]

\[
	H(s) = \frac{V_O}{D'} \cdot \frac{1 - s\frac{L}{R D'^2}}{1 + s\frac{L}{R D'^2} + s^2\frac{L C}{D'^2}}
\]

\clearpage
\section{Bode-Plot}

\begin{figure}[h!]
	\begin{subfigure}[b]{0.9\textwidth}
		\centering
		\begin{tikzpicture}
			\begin{axis}[
				xmode=log,
				ymode=log,
				log basis y=10,
				yticklabel={\pgfmathparse{20*(\tick)}\pgfmathprintnumber[fixed]{\pgfmathresult}},
				height=0.45\textwidth,
				title=\textbf{Power Stage Transfer Function -- Amplitude Response},
				width=\linewidth,
				grid=minor,
				grid style={dashed,gray!30},
				xlabel={angular frequency $\omega \left[\si{\per\second}\right]$},
				ylabel={gain $\left|G\right| \left[\textrm{dB}\right]$},
				x label style={at={(axis description cs:0.5,-0.075)},anchor=north},
				legend style={at={(0.99,0.98)},anchor=north east},
				x tick label style={rotate=90,anchor=east},
				legend columns=1,
				ymin=1,
				ymax=10000,
				xmin=\angflowerbound,
				xmax=\angfupperbound
				]
				\addplot+[mark=none] table[x=angfreq, y=magnitude, col sep=comma]{./../../data/h1.csv};
				\addlegendentry{$H_1$};
				\addplot+[mark=none] table[x=angfreq, y=magnitude, col sep=comma]{./../../data/h2.csv};
				\addlegendentry{$H_2$};
				\addplot+[mark=none] table[x=angfreq, y=magnitude, col sep=comma]{./../../data/h3.csv};
				\addlegendentry{$H_3$};
			\end{axis}
		\end{tikzpicture}
	\end{subfigure}

	\begin{subfigure}[b]{0.9\textwidth}
		\centering
		\begin{tikzpicture}
			\begin{axis}[
				xmode=log,
				ymode=linear,
				height=0.45\textwidth,
				title=\textbf{Power Stage Transfer Function -- Phase Response},
				width=\linewidth,
				grid=minor,
				grid style={dashed,gray!30},
				xlabel={angular frequency $\varphi \left[\si{\per\second}\right]$},
				ylabel={phase $\varphi \left[\si{\degree}\right]$},
				x label style={at={(axis description cs:0.5,-0.075)},anchor=north},
				legend style={at={(0.99,0.98)},anchor=north east},
				x tick label style={rotate=90,anchor=east},
				legend columns=1,
				ymin=\phaselowerbound,
				ymax=\phaseupperbound,
				xmin=\angflowerbound,
				xmax=\angfupperbound
				]
				\addplot+[mark=none] table[x=angfreq, y=phase, col sep=comma]{./../../data/h1.csv};
				\addlegendentry{$H_1$};
				\addplot+[mark=none] table[x=angfreq, y=phase, col sep=comma]{./../../data/h2.csv};
				\addlegendentry{$H_2$};
				\addplot+[mark=none] table[x=angfreq, y=phase, col sep=comma]{./../../data/h3.csv};
				\addlegendentry{$H_3$};
				%\draw[red, dotted, thick] (8,5) -- (9,200);
				\draw[dashed] (axis cs: \angflowerbound,\angfmargin) node[below right]{$\varphi_{R}$ = \SI{-120}{\degree}} -- (axis cs: \angfupperbound,\angfmargin);
			\end{axis}
		\end{tikzpicture}
	\end{subfigure}
\end{figure}
