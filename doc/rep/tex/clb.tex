\section{Closed Loop Behaviour}

\begin{figure}[h!]
	\centering
	\def\svgscale{0.75}
	\input{../fig/sch/control_loop_closed_01.pdf_tex}
	\caption{Closed loop control diagram.}
\end{figure}

\begin{figure}[h!]
	\begin{subfigure}[b]{0.9\textwidth}
		\centering
		\begin{tikzpicture}
			\begin{axis}[
				%xmode=log,
				%ymode=log,
				%log basis y=10,
				%yticklabel={\pgfmathparse{20*(\tick)}\pgfmathprintnumber[fixed]{\pgfmathresult}},
				height=0.9\textwidth,
				title=\textbf{Closed Loop Step Response},
				width=\linewidth,
				grid=major,
				grid style={dashed,gray!30},
				xlabel={time $t \left[\si{\second}\right]$},
				ylabel={amplitude $\left[\si{1}\right]$},
				x label style={at={(axis description cs:0.5,-0.075)},anchor=north},
				legend style={at={(0.98,0.02)},anchor=south east},
				x tick label style={rotate=90,anchor=east},
				legend columns=1,
				ymin=-0.2,
				ymax=1.2,
				xmin=-0.001,
				xmax=0.01
				]
				\addplot+[mark=none] table[x=time, y=signal, col sep=comma]{./../../data/step1.csv};
				\addlegendentry{D = 0.75};
				\addplot+[mark=none] table[x=time, y=signal, col sep=comma]{./../../data/step2.csv};
				\addlegendentry{D = 0.50};
				\addplot+[mark=none] table[x=time, y=signal, col sep=comma]{./../../data/step3.csv};
				\addlegendentry{D = 0.25};
			\end{axis}
		\end{tikzpicture}
	\end{subfigure}

\end{figure}
