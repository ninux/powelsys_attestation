\section{Exercise A -- Steady State Duty Ratio}
To calculate the duty cycle we can use the volt-second balance.
To do so, we have to evaluate the proper mathematical expressions
for the active (first) and passive (second) interval.

\begin{figure}[h!]
	\centering
	\def\svgscale{\schematicscale}
	\input{../fig/sch/booster_complete.pdf_tex}
	\caption{Complete booster circuit.}
\end{figure}

\begin{figure}[h!]
	\centering
	\begin{subfigure}[b]{0.48\textwidth}
		\centering
		\def\svgscale{\schematicscale}
		\input{../fig/sch/booster_first.pdf_tex}
		\caption{First interval.}
	\end{subfigure}
	\begin{subfigure}[b]{0.48\textwidth}
		\centering
		\def\svgscale{\schematicscale}
		\input{../fig/sch/booster_second.pdf_tex}
		\caption{Second interval.}
	\end{subfigure}
	\caption{Equivalent booster circuits for the first and second interval.}
\end{figure}


\subsection{First Interval}
In the active interval the MOSFET is on, hence, conducting the current.
Since the inductance is connected to the negative terminal of the source,
and assuming no voltage drop over the drain-source channel, the voltage
across the inductance $L_1$ is defined by the input voltage $V_i$. 

\[ V_{L_{1}} = V_i \]

%\begin{figure}[h!]
%	\centering
%	\def\svgscale{\schematicscale}
%	\input{../fig/sch/booster_first.pdf_tex}
%	\caption{Equivalent booster circuit for the first interval.}
%\end{figure}

\subsection{Second Interval}
In the passive interval the diode $D$ is conducting the inductor
current to the load and the output capacitor. The capacitor is assumed
to be ideal, hence, no output voltage ripple is assumed. Therefore, the
voltage loop is defined as
\[ V_i = V_{L_{2}} + V_o \]
Using this loop we can obtain the inductors voltage for the passive
interval as
\[ V_{L_{2}} = V_i - V_o \]

%\begin{figure}[h!]
%	\centering
%	\def\svgscale{\schematicscale}
%	\input{../fig/sch/booster_second.pdf_tex}
%	\caption{Equivalent booster circuit for the second interval.}
%\end{figure}

\subsection{Volt-Second Balance}
To calculate the duty cycle of the boost converter we can use the general
volt-second balance equation
\[ V_{L_{1}} D + V_{L_{2}} (1-D) = 0 \]
By rearranging the equation, we can isolate the duty cycle $D$. To do so,
we can expand the factors to get a sum at the left hand side of the equation
\[ V_{L_{1}} D + V_{L_{2}} - V_{L_{2}} D = 0 \]
By separation of the sums we can obtain an expression where the duty cycle $D$
is on the left hand side of the equation
\[ V_{L_{1}} D - V_{L_{2}} D = - V_{L_{2}} \]
Now we can factorize the equation for the duty cycle $D$
\[ D (V_{L_{1}} - V_{L_{2}}) = -V_{L_{2}} \]
Finally we can isolate the duty cycle $D$ by a division of the second factor
\[ D = \frac{- V_{L_{2}}}{V_{L_{1}} - V_{L_{2}}} \]
Using the previously obtained expressions for the inductor voltages of the
first interval $V_{L_{1}}$ and second interval $V_{L_{2}}$ we can express
the duty cycle $D$ as a function of the input voltage $V_i$ and output voltage
$V_o$
\[  D = \frac{-(V_i - V_o)}{V_i - (V_i - V_o)} \]
Expanding this expression leads to the reduced and final expression
\[ D = \frac{- V_i + V_o}{V_i - V_i + V_o} = \frac{- V_i + V_o}{V_o} \]

\subsection{Numerical Results}

\subsubsection{By hand}
\[ V_{i_{1}} = \SI{50}{\volt}  \Rightarrow D = \frac{- \SI{50}{\volt}  + \SI{200}{\volt}}{\SI{200}{\volt}}  = 0.75 \]
\[ V_{i_{1}} = \SI{100}{\volt} \Rightarrow D = \frac{- \SI{100}{\volt} + \SI{200}{\volt}}{\SI{200}{\volt}} = 0.50 \]
\[ V_{i_{1}} = \SI{150}{\volt} \Rightarrow D = \frac{- \SI{150}{\volt} + \SI{200}{\volt}}{\SI{200}{\volt}} = 0.25 \]

\subsubsection{By software}
\lstinputlisting{../../data/duty.csv}

%\clearpage
%\subsubsection{GNU Octave Code}
\lstinputlisting{../../sim/boostduty.m}
%\lstinputlisting{../../sim/attestation.m}
\clearpage
