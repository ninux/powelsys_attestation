\section{Exercise A -- Steady State Duty Ratio}
To calculate the duty cycle we can use the volt-second balance.
To do so, we have to evaluate the proper mathematical formulation
for the active (first) and passive (second) interval.

\begin{figure}[h!]
	\centering
	\def\svgscale{\schematicscale}
	\input{../fig/sch/booster_complete.pdf_tex}
	\caption{Complete booster circuit.}
\end{figure}

\subsection{First Interval}

\begin{figure}[h!]
	\centering
	\def\svgscale{\schematicscale}
	\input{../fig/sch/booster_first.pdf_tex}
	\caption{Equivalent booster circuit for the first interval.}
\end{figure}

In the active interval the MOSFET is on. Since the inductance is
connected to the negative terminal of the source the voltage across
the inductance $L_1$ is the input voltage $V_i$. 
\[ V_{L_{1}} = V_i \]

\subsection{Second Interval}

\begin{figure}[h!]
	\centering
	\def\svgscale{\schematicscale}
	\input{../fig/sch/booster_second.pdf_tex}
	\caption{Equivalent booster circuit for the second interval.}
\end{figure}

In the passive interval the diode $D$ is conducting the inductor
current to the load and the output capacitor. The capacitor is assumed
to be ideal, hence no output voltage ripple is assumed. Therefore, the
voltage loop is defined as
\[ V_i = V_{L_{2}} + V_o \]
Using this loop we can obtain the inductors voltage for the passive
interval as
\[ V_{L_{2}} = V_i - V_o \]

\subsection{Volt-Second Balance}
To calculate the duty cycle we can use the general volt-second balance
equation
\[ V_{L_{1}} D + V_{L_{2}} (1-D) = 0 \]
and rearrange the equation so isolate the duty cycle $D$. Herefore
we can expand the factors to
\[ V_{L_{1}} D + V_{L_{2}} - V_{L_{2}} D = 0 \]
and separate the sums
\[ V_{L_{1}} D - V_{L_{2}} D = - V_{L_{2}} \]
and factorize for
\[ D (V_{L_{1}} - V_{L_{2}}) = V_{L_{2}} \]
and finally isolate 
\[ D = \frac{- V_{L_{2}}}{V_{L_{1}} - V_{L_{2}}} \]

Using the previously obtained expressions for $V_{L_{1}}$ and
$V_{L_{2}}$ we get
\[  D = \frac{-(V_i - V_o)}{V_i - (V_i - V_o)} \]
Expanding this expression leads to
\[ D = \frac{- V_i + V_o}{V_i - V_i + V_o} = \frac{- V_i + V_o}{V_o} \]

\subsection{Numerical Results}

\subsubsection{By Hand}
\[ V_{i_{1}} = \SI{50}{\volt}  \Rightarrow D = \frac{- \SI{50}{\volt}  + \SI{200}{\volt}}{\SI{200}{\volt}}  = 0.75 \]
\[ V_{i_{1}} = \SI{100}{\volt} \Rightarrow D = \frac{- \SI{100}{\volt} + \SI{200}{\volt}}{\SI{200}{\volt}} = 0.50 \]
\[ V_{i_{1}} = \SI{150}{\volt} \Rightarrow D = \frac{- \SI{150}{\volt} + \SI{200}{\volt}}{\SI{200}{\volt}} = 0.25 \]

\subsubsection{With GNU Octave}
\lstinputlisting{../../data/duty.csv}

%\clearpage
%\subsection{GNU Octave Code}
%\lstinputlisting{../../sim/boostduty.m}
%\lstinputlisting{../../sim/attestation.m}
%\clearpage
