\thispagestyle{empty}
\otherlanguage{english}
\begin{abstract}
	This report shows the individual solution of exercise 5.5 from the module
	Power Electronic Systems (TSM\_PowElSys), which is neccessary for admission
	to the final exam.

	The report shows how a boost converter can be analysed for the frequency
	domain using the state space averaging, hence, deriving the transfer
	function of the converter. Further, the report shows how a controller can
	be designed for the boost converter.

	The developed transfer functions of the power stage, the controller,
	and the closed loop behaviour is shown with simulation results using
	GNU Octave.
\end{abstract}
%%------------------------------------------------------------------------------
%% Hintergrund
%%------------------------------------------------------------------------------
%%\todo{HINTERGRUND}
%% Was ist bereits bekannt zum Thema?
%%\todo{Was ist bereits bekannt zum Thema?}
%The use of batteries in emergency supplies demands a diagnose which estimates
%the state of charge and state of health of the battery. The use of a reliable
%diagnosis of the battery can ensure the correct operation and optimal use of
%batteries' service life.
%
%% Was ist noch unbekannt?
%%\todo{Was ist noch unbekannt?}
%This study examines which methods are suitable for the diagnosis of
%lead-acid batteries used in emergency supplies of elevators.
%%
%% Was soll die Arbeit untersuchen?
%%\todo{Was soll die Arbeit untersuchen?}
%It aims to shows how the existing application can be extended with a
%method for diagnosis. Hence, a suitable diagnosis system is developed and
%implemented. Additionally, the study explores the requirements for the
%existing system. Furthermore, it aims to show which modifications and future
%work should be addressed for the application, along with a possible
%optimization of the diagnosis system.
%
%%------------------------------------------------------------------------------
%% Methode 
%%------------------------------------------------------------------------------
%
%% Wie war das Vorgehen?
%%\todo{Wie war das vorgehen?}
%After the exploration of relevant literature, existing methods were classified
%and rated. The 2-pulse load test was evaluated as the most suitable method to
%be applied. This method was validated through datasheet study of the batteries
%and laboratory tests. Based on the analysis of the existing battery management
%system, a new diagnosis system was developed. A prototype of this new diagnosis
%system has been developed, integrated, and tested in the existing application.
%
%%	
%% Welche Mittel sind verwendet worden?
%%\todo{Welche Mittel sind verwendet worden?}
%%Die Validierung des 2-Puls Lasttest erfolgte mit einem Testsystem, welches das
%%Verfahren auf unterschiedliche Batterietypen und mit unterschiedlichen
%%Testparametern durchführte.
%For the validation the 2-pulse load test was applied within a test installation
%and checked with coulomb counting measurements. The test was performed with
%different batteries which are used in the existing application. The test of the
%implemented prototype was performed on a reduced setup which simulates the use
%in the real application with controlled conditions.
%
%%	
%% Unter welchen Bedingungen ist untersucht worden?
%%\todo{Unter welchen Bedingungen ist untersucht worden?}
%%For the validation the 2-pulse load test was applied on different types of
%%batteries with multiple samples, different C-rates, multiple temperatures
%%and different state of charge of the batteries.
%%The validation was performed with
%
%%------------------------------------------------------------------------------
%% Ergebnisse
%%------------------------------------------------------------------------------
%%\todo{ERGEBNISSE}
%%The study shows that the 2-pulse load test is suitable for the diagnosis of
%%lead-acid batteries in emergency supplies of elevators and a diagnosis system
%%can be integrated in the existing applications.
%
%The validation shows that the state of charge and state of health can be
%estimated with the 2-pulse load test. The required models and parameters have
%been identified. Optimal estimations were observed with the use of specific
%battery models. The estimation of the state of health is only reliable for
%fully charged batteries and improves with higher load pulses.
%%
%For the use of the 2-pulse load test a new diagnosis system is introduced and
%a prototype integrated in the existing application. The tests of the implemented
%prototype shows that the 2-pulse load test can be integrated in the existing
%application.
%
%%------------------------------------------------------------------------------
%% Schlussfolgerung
%%------------------------------------------------------------------------------
%%\todo{SCHLUSSFOLGERUNGEN}
%There is no standard method for the diagnosis of batteries in emergency
%supplies. For these applications the 2-pulse load test is suitable as a
%method for the diagnosis of batteries. The test delivers an estimation of the
%state of charge and state of health of the battery within less than 8 minutes
%and without an interruption of the emergency supply availability. The state of
%health estimations reaches an accuracy of at least $\pm 15\%$ at ideal conditions.
%
%% Hauptergebnis
%%\todo{Hauptergebnis}
%
%% Zusätzliche Feststellungen
%%\todo{Zusätzliche Feststellungen}
%
%% Perspektiven
%%\todo{Perspektiven}
%The results presented in this study have to be extended with further
%investigations. Especially the implemented diagnosis system has to be tested
%for long time operation with different batteries. Further, a strategy has to
%be developed, how to use the diagnosis and how the estimations are processed.
%\end{abstract}
%\otherlanguage{ngerman}
