% https://www.ncbi.nlm.nih.gov/pmc/articles/PMC3136027/
\thispagestyle{empty}
\begin{abstract}
%------------------------------------------------------------------------------
% Hintergrund
%------------------------------------------------------------------------------
%\todo{HINTERGRUND}
% Was ist bereits bekannt zum Thema?
%\todo{Was ist bereits bekannt zum Thema?}
Der Einsatz von Batterien in Notstromversorgungen bedarf einer
Diagnose, welche den Gesundheits- und Ladezustand der Batterie schätzt.
Durch eine zuverlässige Diagnose der Batterie kann der korrekte Betrieb
gewährleistet und die Lebensdauer der Batterie optimal genutzt werden.

% Was ist noch unbekannt?
%\todo{Was ist noch unbekannt?}
Die vorliegende Arbeit untersucht, welche Verfahren sich für die Diagnose
von Bleibatterien eignen, welche in der Notstromversorgung von Aufzugsanlagen
eingesetzt werden.
%
% Was soll die Arbeit untersuchen?
%\todo{Was soll die Arbeit untersuchen?}
Die Arbeit soll zeigen, wie die bestehende Anwendung um ein
Diagnosverfahren erweitert werden kann. Hierzu soll ein entsprechendes
Diagnosesystem entworfen und als Prototyp in die bestehende Anwendung
implementiert werden. Zusätzlich ist zu untersuchen, welche Massnahmen
für den Einsatz und die Optimierung des Diagnosesystems vorzusehen sind.
	
%------------------------------------------------------------------------------
% Methode 
%------------------------------------------------------------------------------

% Wie war das Vorgehen?
%\todo{Wie war das vorgehen?}
Nach Sichtung der relevanten Literatur wurden die Verfahren kategorisiert
und bewertet. Dabei wurde für die vorliegende Anwendung der 2-Puls Lasttest
als das am besten geeignete Verfahren identifiziert. Dieser wurde mittels
Untersuchungen von Datenblattangaben und Labortests validiert. Ausgehend
von einer Analyse des bestehenden Batteriemanagementsystems wurde ein
neues Diagnosesystem entworfen. Dieses neue Diagnosesystem wurde als
Prototyp in die bestehende Anwendung integriert und getestet.

%
% Welche Mittel sind verwendet worden?
%\todo{Welche Mittel sind verwendet worden?}
Für die Validierung wurde der 2-Puls Lasttest in einem Testsystem durchgeführt
und mittels Ladungsmessungen überprüft.
Hierbei wurden unterschiedliche Batterien eingesetzt, welche in der
vorliegenden Anwendung verwendet werden.
Die Tests am realisierten Prototypen erfolgten in einer reduzierten
Installation, welche den realen Einsatz unter kontrollierten Bedingungen
simuliert.
	
%	
% Unter welchen Bedingungen ist untersucht worden?
%\todo{Unter welchen Bedingungen ist untersucht worden?}
%Die Validierung des 2-Puls Lasttest erfolgte in einer Temperaturkammer.
%Dabei wurde der 2-Puls Lasttest mit unterschiedlichen C-Raten, Ladezuständen
%und Temperaturen untersucht. Die Tests des Prototypen erfolgten in einem
%Testraum und unter Einsatz einer gewählten C-Rate.
	
%------------------------------------------------------------------------------
% Ergebnisse
%------------------------------------------------------------------------------
%\todo{ERGEBNISSE}
Die durchgeführte Validierung zeigt, dass der Ladezustand und der Gesundheitszustand
mit dem 2-Puls Lasttest geschätzt werden kann. Die hierzu notwendigen Modelle
und Parameter konnten identifiziert werden. Optimale Schätzungen werden dabei mit
Anwendung spezifischer Batteriemodelle erzielt. Die Schätzung des Gesundheitszustandes
ist ausschliesslich bei vollgeladener Batterie zuverlässig und verbessert sich
mit höheren Lastpulsen. Für den Einsatz des 2-Puls Lasttest wird ein Entwurf eines
neuen Diagnosesystems vorgestellt und als Prototyp in die bestehende Anwendung
realisiert. Die Tests des realisierten Prototypen zeigen, dass der 2-Puls Lasttest
in der vorliegenden Anwendung integriert werden kann.


%Die Untersuchungen zeigen, dass sich der 2-Puls Lasttest für die Diagnose von
%Bleibatterien in Notstromversorgungen von Aufzugsanlagen eignet. Mit diesem
%kann jederzeit der Ladezustand und während der Erhaltungsladung auch der
%Gesundheitszustand der Batterie geschätzt werden. Für den Einsatz des
%2-Puls Lasttest wird ein Diagnosesystem vorgestellt, welches in das
%bestehende Batteriemanagementsystem integriert werden kann. Die Tests des
%realisierten Prototypen belegen, dass der 2-Puls Lasttest in der besthenden
%Anwendung realisiert werden kann.

%------------------------------------------------------------------------------
% Schlussfolgerung
%------------------------------------------------------------------------------
%\todo{SCHLUSSFOLGERUNGEN}
Für die Diagnose von Batterien in Notstromversorgungen existiert kein
Standardverfahren. Für die vorliegende Anwendungen bietet sich der 2-Puls
Lasttest zur Diagnose der Batterien an. Dieser liefert innert 8
Minuten eine Schätzung des Ladezustandes und des Gesundheitszustandes, ohne
dabei die Verfügbarkeit der Notstromversorgung zu unterbrechen. Die
Schätzung des Gesundheitszustandes erreicht dabei im Idealfall eine
Genauigkeit von mindestens $\pm 15\%$.

% Hauptergebnis
%\todo{Hauptergebnis}
%Der 2-Puls Lasttest eignet sich als Verfahren zur Schätzung des Ladezustandes
%als auch des Gesundheitszutandes einer Batterie und kann in der
%vorliegenden Anwendung umsetzt werden. Die Schätzung des Gesundheitszustandes
%erzielt dabei unter idealen Bedingungen eine Genauigkeit von mindestens
%$\pm15\%$.

% Zusätzliche Feststellungen
%\todo{Zusätzliche Feststellungen}

% Perspektiven
%\todo{Perspektiven}
Der realisierte Prototyp des Diagnosesystems ist in zukünftigen
Arbeiten fertigzustellen und einem Dauertest mit unterschiedlichen
Batterien zu unterziehen. Weiter ist eine Strategie zu definieren,
wie die Diagnose zu verwenden ist und wie die Schätzwerte
weiterverarbeitet werden.

%Die in der Arbeit dargelegten Ergebnisse sind durch weitere Untersuchungen
%zu erweitern. Insbesondere ist das implementierte Diagnosesystem einer
%langzeitlichen Untersuchung zu unterziehen, welche die verschiedenen
%Betriebsszenarien der realen Anwendung überprüft. Ebenso sind weitere
%Untersuchungen durchzuführen, welche die Anwendung des 2-Puls Lasttest
%mit Batterien überprüft, welche einen Gesundheitszustand zwischen
%neuwertigen und weit degradierten Batterien aufweisen.
\end{abstract}
